\documentclass[]{article}
\usepackage{lmodern}
\usepackage{amssymb,amsmath}
\usepackage{ifxetex,ifluatex}
\usepackage{fixltx2e} % provides \textsubscript
\ifnum 0\ifxetex 1\fi\ifluatex 1\fi=0 % if pdftex
  \usepackage[T1]{fontenc}
  \usepackage[utf8]{inputenc}
\else % if luatex or xelatex
  \ifxetex
    \usepackage{mathspec}
  \else
    \usepackage{fontspec}
  \fi
  \defaultfontfeatures{Ligatures=TeX,Scale=MatchLowercase}
\fi
% use upquote if available, for straight quotes in verbatim environments
\IfFileExists{upquote.sty}{\usepackage{upquote}}{}
% use microtype if available
\IfFileExists{microtype.sty}{%
\usepackage{microtype}
\UseMicrotypeSet[protrusion]{basicmath} % disable protrusion for tt fonts
}{}
\usepackage[margin=1in]{geometry}
\usepackage{hyperref}
\hypersetup{unicode=true,
            pdftitle={proposal\_1},
            pdfauthor={Christina Lyu},
            pdfborder={0 0 0},
            breaklinks=true}
\urlstyle{same}  % don't use monospace font for urls
\usepackage{graphicx,grffile}
\makeatletter
\def\maxwidth{\ifdim\Gin@nat@width>\linewidth\linewidth\else\Gin@nat@width\fi}
\def\maxheight{\ifdim\Gin@nat@height>\textheight\textheight\else\Gin@nat@height\fi}
\makeatother
% Scale images if necessary, so that they will not overflow the page
% margins by default, and it is still possible to overwrite the defaults
% using explicit options in \includegraphics[width, height, ...]{}
\setkeys{Gin}{width=\maxwidth,height=\maxheight,keepaspectratio}
\IfFileExists{parskip.sty}{%
\usepackage{parskip}
}{% else
\setlength{\parindent}{0pt}
\setlength{\parskip}{6pt plus 2pt minus 1pt}
}
\setlength{\emergencystretch}{3em}  % prevent overfull lines
\providecommand{\tightlist}{%
  \setlength{\itemsep}{0pt}\setlength{\parskip}{0pt}}
\setcounter{secnumdepth}{0}
% Redefines (sub)paragraphs to behave more like sections
\ifx\paragraph\undefined\else
\let\oldparagraph\paragraph
\renewcommand{\paragraph}[1]{\oldparagraph{#1}\mbox{}}
\fi
\ifx\subparagraph\undefined\else
\let\oldsubparagraph\subparagraph
\renewcommand{\subparagraph}[1]{\oldsubparagraph{#1}\mbox{}}
\fi

%%% Use protect on footnotes to avoid problems with footnotes in titles
\let\rmarkdownfootnote\footnote%
\def\footnote{\protect\rmarkdownfootnote}

%%% Change title format to be more compact
\usepackage{titling}

% Create subtitle command for use in maketitle
\newcommand{\subtitle}[1]{
  \posttitle{
    \begin{center}\large#1\end{center}
    }
}

\setlength{\droptitle}{-2em}

  \title{proposal\_1}
    \pretitle{\vspace{\droptitle}\centering\huge}
  \posttitle{\par}
    \author{Christina Lyu}
    \preauthor{\centering\large\emph}
  \postauthor{\par}
      \predate{\centering\large\emph}
  \postdate{\par}
    \date{10/2/2018}


\begin{document}
\maketitle

\section{Group Members:}\label{group-members}

Christina Lyu, Aoi, Ahlam

\section{Title:}\label{title}

Relationship between Salaries after Graduation and Race, Major or Gender

\section{Purpose:}\label{purpose}

As college students, we work hard for our education and it is true to
almost everyone that we would want to get a job one time or another.
Therefore, we want to look at the factors that would affect ones salary.
For example, it's very common that the people with computer science
degrees might get jobs easier than people with art degrees. It is also
common that people with higher degrees get higher salaries than people
in the same field but with lower degrees. But majors aren't the only one
with an influence. Despite the fact that we try to build a world with no
discriminations, there still exist in the society different treatments
for different people so we are also interested to see how races, gender
would make change to the salaries.

\section{Data:}\label{data}

1.Our data was found on this website. Each row represent a specific
major in a specific year, with columns representing different factors
including teaching, student, male, white and so on. The numbers are the
average yearly salaries for that factor. There lines with all 0s means
that major was not created that year.
\url{https://think.cs.vt.edu/corgis/csv/graduates/graduates.csv?forcedownload=1}

2.The data has multiple categories that could have played a role in the
salary of the individual, rather than just the major i.e.~year,
minorities, gender, etc.

\url{https://catalog.data.gov/dataset/current-employee-names-salaries-and-position-titles-840f7/resource/2b09fa08-c85e-4bbe-a90d-cdc4ae39b0cc}

3.The data compares the salary for people who live in the city in
Chicago. A flaw in us using this data is that this data is catered to
Chicago, so we should be more specific if we want data for a specific
city or just the national average.

\url{https://cew.georgetown.edu/cew-reports/valueofcollegemajors/\#explore-data}

4.This data is in response to the previous data presented. This specific
data shows earning by state and includes the percentiles for the
earnings. I think if we wanted to do a box-and-whisker plot for each
state to show the data, it would be great to use the information from
the data.

\url{https://www.forbes.com/sites/susanadams/2015/07/02/the-college-majors-with-the-highest-starting-salaries/\#ed527cb35024}

5.This data shows that our hypothesis was correct in which STEM majors
make more than Humanities majors.
\url{http://online.wsj.com/public/resources/documents/info-Degrees_that_Pay_you_Back-sort.html}

6.This data shows a more extensive version of college majors, and it
also shows the different percentiles for the salaries. With these
percentiles, we can definitely compare the different majors and see if
there are any significant differences.

\url{https://www.glassdoor.com/blog/50-highest-paying-college-majors/}

7.This data only focuses on the median salary, which is different than
the previous dataset. This would show more about the middle 50\% of
salaries, which would be good to not have any extreme values.

\url{https://www.cnbc.com/2016/10/19/the-50-highest-paying-college-majors.html}

8.The data shows the top 50 highest paying college majors, which would
not include every major. This may or may not include certain categories
in our hypothesis.

9.The data below shows the expected salaries for undergraduates who
majored in humanities and arts. It listed philosophy as the top paid
major in the class of 2016 out of the other humanities and arts majors.

\url{http://www.naceweb.org/job-market/compensation/philosophy-projected-as-top-paid-class-of-2016-humanities-major/}

With all these differing data, I think we should narrow down the results
to see if we want it to be by state or nationwide.

\section{Population:}\label{population}

The population of the study is everyone with a college degree and a job.
Since there are columns called ``Bachelors'' and ``Masters'', which
means we are taking samples with someone with at least a bachelors
degree. The values under the columns represent the salaries one get so
our sample must have jobs. The observation units are majors in different
years. I'll try to generalize the predicted salary for people after they
graduate based on their gender, race, major, degree and so on. There are
517 rows of data in the data frame, but I'm sure there are majors that
are not covered in the data.

\section{Response Variable:}\label{response-variable}

The response variable is the salary measured in units Dollar and has an
approximate range of 1000 to 200,000.

\section{Explanatory Variables:}\label{explanatory-variables}

The explanatory variables can be many possible combinations including
Student, Teaching, Whites, Male, Major, Bachelors and so on. We don't
know which of them are statistically significant yet. They are all
categorical variables so the actual values are 0 or 1. But in the data
set, each element will have a corresponding salary as the mean salary
for someone with the major that year who has a value of 1 for that
factor.


\end{document}
